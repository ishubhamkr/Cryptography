\documentclass{article}
\usepackage[utf8]{inputenc}
\usepackage{graphicx}
\graphicspath{ {./images/} }
\newcommand\tab[1][3cm]{\hspace*{#1}}
\newtheorem{defn}{Definition} [section]
\newtheorem{ex}{Example}[section]
\usepackage{listings}
\usepackage{xcolor} % for setting colors
\lstset{
    frame=tb, % draw a frame at the top and bottom of the code block
    tabsize=4, % tab space width
    showstringspaces=false, % don't mark spaces in strings
    numbers=left, % display line numbers on the left
    commentstyle=\color{green}, % comment color
    keywordstyle=\color{blue}, % keyword color
    stringstyle=\color{red} % string color
}
\title{"Basics of Crptography"}
\author{Shubham Kumar\thanks{Mtech-1st yr: Information Security, Atal Bihari Vajpayee Indian Institute of Information Technology and Management, Gwalior, India-474015} }
\date{October 2020}

\begin{document}

\maketitle


\section{Set Theory:-}
\begin{itemize}
    \item Universal Set or Universe of discourse: Collection of Objects.
	\item Set: Collection of well-defined objects. Here the term well defined refers, the definition of a set is not person dependent.
	\item It can be defined using Characteristic function. This tells us that the entire (any) Mathematical structure is built on the binary logic.
	\item Cartesian product of two sets $A$ and $B$ is nothing but the set of all possible combinations of elements of $A$ and $B$
	\item Function: Every element of domain is associated with some other (unique) element of the co-domain
	\item Binary Operator: $*$ is said to be a binary operator and is defined as $*:G \times G \rightarrow G $
	\item Arithmetic Operation: Addition, Subtraction, Multiplication, Division (all are binary operators on $\Re$)
\end{itemize}
\section{Group Theory:-}
\begin{defn}
A set $G$ with a Binary operation $*$ defined on $G$ is said to be a {\textbf{Group}}, if it satisfies the following four axioms,
\begin{itemize}
	\item Closure: $$\forall a, b \in G \Rightarrow a*b \in G$$
	
	\item Associative: $$a*(b*c)=(a*b)*c, \forall a,b,c \in G$$
	
	\item Identity: For every $a \in G$, there exists a unique element $e$ such that, $a*e=e*a=a$, then 'e' is called as identity element
	\item Inverse: For every non-zero element $a$ of $G$, there exists a unique non-zero $a' \in G$ such that, $$a*a'=a'*a=e$$
	\item Commutative: $$a*b=b*a, \forall a,b \in G$$\\
	If a Group satisfies the Commutative property, then it is called as an Abelian Group.
\end{itemize}
\subsection{Classification of a Algebraic Structure}
\begin{itemize}
    \item Groupoid: Closure
    \item SemiGroup: Closure + Associative
    \item Monoid: Closure + Associative + Identity
    \item Group: Closure + Associative + Identity + Inverse
    \item Abelian Group: If a Group satisfies the Commutative property.
\end{itemize}
\subsection{Properties of Group}
\begin{itemize}
    \item If a group has a finite number of elements, it is referred to as a finite group, and the order of the group is equal to the number of elements in the group. Otherwise, the group is an infinite group.
    \item Identity Element($e$) is unique for a Group.
    \item Left Identity Should be same as Right Identity : $$a*e=e*a$$
    \item $a^{-1}$ is unique for a given element a.
    \item $(a*b)^{-1}=b^{-1}*a^{-1}$
    \item If G is Abelian Group than: $$(a*b)^{-1}=b^{-1}*a^{-1}$$ $$(a*b)^{-1}=a^{-1}*b^{-1}$$
\end{itemize}
\subsubsection{Cyclic Group}
\begin{itemize}
    \item A group G is cyclic if every element of G is a power $a^{k}$ (k is an integer) of a fixed element $a \in G$. The element a is said to generate the group G, or to be a generator of G.We denote the cyclic group of order n by $Z_n$.
    \item A cyclic group is always an abelian group, and may be finite or infinite but and abelian group need not be cyclic group.
    \item A non-abelian group will always be non-cyclic.
    
\end{itemize}
\begin{ex} Check the following (Whether Cyclic or not)
\\For $n \geq 1$ $({Z}_n,\oplus)$ is cyclic
\\Solution: $Z_n$ is an abelian group
\\ For $\oplus$, $Z_n$ generates number $a_i$ where $a_i < n$
\\ Then we can find a subgroup of $Z_n$ which generates a repetative group. So, we can say that $Z_n$ is cyclic group.
\end{ex}
\begin{ex}Check the following (whether Group or not)
\begin{enumerate}
	\item The set of integers with usual addition 
	\\ Solution:- Let $a,b,c \in Z$, then 
	\\Closure Property: $$a+b \in Z$$
	\\Associative Property: $$a+(b+c)=(a+b)+c, \forall a,b,c \in Z$$
	\\Identity Property: $$a+0=0+a=a, \forall a \in Z$$
	\\Inverse Property: $$ \ a + \ (-a\ ) =\ (-a\ ) + a = 0, \forall a \in Z$$
	\\Commutative Property: $$a+b=b+a,\forall a,b \in Z$$
	\\So, the set of integers with usual addition is not only group but it's an abelian group.\\
	\item The set of integers with usual multiplication
	\\ Solution:- Let $a,b,c \in Z$, then 
	\\Closure Property: $$a*b \in Z$$
	\\Associative Property: $$a*(b*c)=(a*b)*c, \forall a,b,c \in Z$$
	\\Identity Property: $$a*1=1*a=a, \forall a \in Z$$
	\\Inverse Property: $$a*a^{-1}= a^{-1} * a = 1 , \forall a \in Z$$
	\\It doesn't holds because except for 1, inverse of every number is not integer it's in fraction.
	\\Therefore, the set of integers under multiplication is not a group.\\
	\item The set of real numbers with usual addition
	\\Solution:- Let $a,b,c \in R$, then 
	\\Closure Property: $$a+b \in R$$
	\\Associative Property: $$a+(b+c)=(a+b)+c, \forall a,b,c \in R$$
	\\Identity Property: $$a+0=0+a=a, \forall a \in R$$
	\\Inverse Property: $$ \ a + \ (-a\ ) =\ (-a\ ) + a = 0 , \forall a \in R$$
	\\Commutative Property: $$a+b=b+a,\forall a,b \in R$$
	\\So, the set of real numbers with usual addition is not only group but it's an abelian group.\\
	\item The set of real numbers with usual multiplication
	\\Solution:- Let $a,b,c \in R$, then 
	\\Closure Property: $$a*b \in R$$
	\\Associative Property: $$a*(b*c)=(a*b)*c, \forall a,b,c \in R$$
	\\Identity Property: $$a*1=1*a=a, \forall a \in R$$
	\\Inverse Property: $$a*a^{-1}= a^{-1} * a = 1 , \forall a \in R$$
	\\Commutative Property: $$a+b=b+a,\forall a,b \in R$$
	\\Therefore, the set of real numbers under multiplication is not only group but it's an abelian group.\\
	\item The set of natural numbers with usual addition
	\\Solution:- Let $a,b,c \in N$, then 
	\\Closure Property: $$a+b \in N$$
	\\Associative Property: $$a+(b+c)=(a+b)+c, \forall a,b,c \in N$$
	\\Identity Property: $$a+0=0+a=a, \forall a \in N$$
	\\Inverse Property: $$ \ a + \ (-a\ ) =\ (-a\ ) + a = 0 , \forall a \in R, but except for a = 0, -a \notin N $$
	\\So, The set of natural numbers under addition is not a group beause it does not have the inverse property .\\
	\item $A=\{0,1,2,3\}, a*b=a+b-ab$
	\\Closure Property: For a=3 and b=3
	\\ $$ a*b=a+b-ab = 3+3-(3*3) $$
    \tab	$a*b= 6-9 = -3$ and $-3 \notin A$ 
    \\So,it is not a group.
\end{enumerate}
\end{ex}
\end{defn}
\section{Rings:-}
\begin{defn} A set $R$ with two Binary operations $*_1$ and $*_2$ (denoted by \textbf{$(R, *_1,*_2)$}) is said to be a \emph{\textbf{Ring}}, if
It satisfies the following axioms,\\
\begin{itemize}
    \item $(R,*_1)$ is an abelian group
	\item Closure: $$\forall a, b \in R \Rightarrow a*_2b \in R$$
	
	\item Associative: $$a*_2(b*_2c)=(a*_2b)*_2c, \forall a,b,c \in R$$
	
	\item Identity: For any $a \in R$, there exists a unique $e_2 \in R$ such that, $$a*_2e_2=e_2*_2a=a$$, then '$e_2$' is called as an identity element of $R$ w.r.t. $*_2$.
		\item Distributive of $*_1$ over $*_2$:\\
	$a*_2(b*_1c)=(a*_2b)*_1 (a*_2c)$\\
	$(a*_1b)*_2c=(a*_1c)*_2 (b*_1c)$\\
	\\ When it satisfies the above mention properties, then it is called ring.
	\item Commutative: $$a*_2b=b*_2a, \forall a,b \in G$$ When a Ring satisfies commutative property w.r.t. $*_2$, then we call $(R,*_1,*_2)$ as a \textbf{\emph{commutative ring with identity}}
\subsection{Integral Domain}
Commutative ring that obeys the following axioms
\begin{itemize}
    \item Multiplicative identity: There is an element 1 in R such that a1=1a=a, $ \forall a \in R $
    \item Non Zero divisors: If a, b in R and ab = 0, then either a = 0 or b=0.
\end{itemize}
\end{itemize}
\begin{ex} Check the following (Whether Ring or not)
\begin{enumerate} 
	\item The set of integers with usual addition and multiplication
	\\ Solution:- Set of integers with usual addition is an abelian group as shown above in example (2.1-1)
	\\For multiplication:
	\begin{itemize}
	    \item Closure: $$\forall a, b \in Z \Rightarrow a*b \in Z$$
        \item Associative: $$a*(b*c)=(a*b)*c, \forall a,b,c \in Z$$
	    \item Identity: For any $a \in Z$, there exists a unique $e_2 \in R$ such that, $$a*1=1*a=a$$, then '$1$' is an identity element of $Z$ w.r.t. $*$.
		\item Distributive of $+$ over $*$:\\
	        $a*(b+c)=(a*b)+ (a*c)$  $ \forall a,b,c \in Z $ \\
	        $(a+b)*c=(a*c)+(b*c)$   $ \forall a,b,c \in Z $ \\
	\\Therefore the set of integers with usual addition and multiplication is a ring.
	\end{itemize}
	\item The set of real numbers with usual addition and multiplication
	\\ Solution:- Set of real numbers with usual addition is an abelian group as shown above in example (2.1-3)
	\\For multiplication:
	\begin{itemize}
	    \item Closure: $$\forall a, b \in R \Rightarrow a*b \in R$$
        \item Associative: $$a*(b*c)=(a*b)*c, \forall a,b,c \in R$$
	    \item Identity: For any $a \in R$, $1$ is an identity element of $R$ such that, $$a*1=1*a=a$$ .
		\item Distributive of $+$ over $*$:\\
	        $a*(b+c)=(a*b)+ (a*c)$  $ \forall a,b,c \in R $ \\
	        $(a+b)*c=(a*c)+(b*c)$   $ \forall a,b,c \in R $ \\
	\\Therefore the set of real numbers with usual addition and multiplication is a ring.
	\end{itemize}
	\item The set of rational numbers with usual addition and multiplication
	\\ Solution:- Let $a,b,c \in Q$, then 
	\\Closure Property: $$a+b \in Q$$
	\\Associative Property: $$a+(b+c)=(a+b)+c, \forall a,b,c \in Q$$
	\\Identity Property: $$a+0=0+a=a, \forall a \in Q$$
	\\Inverse Property: $$ \ a + \ (-a\ ) =\ (-a\ ) + a = 0, \forall a \in Q$$
	\\Commutative Property: $$a+b=b+a,\forall a,b \in Q$$
	\\So, the set of rational numbers with usual addition is an abelian group.\\
	\\For multiplication:
	\begin{itemize}
	    \item Closure: $$\forall a, b \in R \Rightarrow a*b \in Q$$
        \item Associative: $$a*(b*c)=(a*b)*c, \forall a,b,c \in Q$$
	    \item Identity: For any $a \in Q$, $1$ is an identity element of $Q$ such that, $$a*1=1*a=a$$ .
		\item Distributive of $+$ over $*$:\\
	        $a*(b+c)=(a*b)+ (a*c)$  $ \forall a,b,c \in Q $ \\
	        $(a+b)*c=(a*c)+(b*c)$   $ \forall a,b,c \in Q $ \\
	    \\Therefore the set of rational numbers with usual addition and multiplication is a ring.
	\end{itemize}
	\item The set of Even integers(2N) with usual addition and multiplication
	\\ Solution:- 	\\Closure Property: $$a+b \in 2N$$
	\\Associative Property: $$a+(b+c)=(a+b)+c, \forall a,b,c \in 2N$$
	\\Identity Property: $$a+0=0+a=a, \forall a \in 2N$$
	\\Inverse Property: $$ \ a + \ (-a\ ) =\ (-a\ ) + a = 0, \forall a \in 2N$$
	\\Commutative Property: $$a+b=b+a,\forall a,b \in 2N$$
	\\So, the set of even integers with usual addition is an abelian group.\\
	\\For multiplication:
	\begin{itemize}
	    \item Closure: $$\forall a, b \in R \Rightarrow a*b \in 2N$$
        \item Associative: $$a*(b*c)=(a*b)*c, \forall a,b,c \in 2N$$
	    \item Identity: For any $a \in Q$, $1$ is an identity element of $2N$ such that, $$a*1=1*a=a$$ .
		\item Distributive of $+$ over $*$:\\
	        $a*(b+c)=(a*b)+ (a*c)$  $ \forall a,b,c \in 2N $ \\
	        $(a+b)*c=(a*c)+(b*c)$   $ \forall a,b,c \in 2N $ \\
	    \\Therefore the set of even integers with usual addition and multiplication is a ring.
    \end{itemize}	
\end{enumerate}
\end{ex}
\end{defn}
\section{Fields:-}
\begin{defn} A set $F$ with two Binary operations $*_1,*_2$ (denoted by $(F,*_1,*_2)$) is said to be a field, if it satisfies the following axioms,
\begin{enumerate}
	\item $(F,*_1,*_2)$ is a commutative ring with identity 
	\begin{enumerate}
		\item $(F,*_1)$ is an abelian group
	\item Closure: $$\forall a, b \in F \Rightarrow a*_2b \in F$$
	
	\item Associative: $$a*_2(b*_2c)=(a*_2b)*_2c, \forall a,b,c \in F$$
	
	\item Identity: For any $a \in F$, there exists a unique $e_2 \in F$ such that, $$a*_2e_2=e_2*_2a=a$$, then '$e_2$' is called as an identity element of $F$ w.r.t. $*_2$.
		\item Distributive of $*_1$ over $*_2$:\\
	$a*_2(b*_1c)=(a*_2b)*_1 (a*_2c)$\\
	$(a*_1b)*_2c=(a*_1c)*_2 (b*_1c)$
	\item Commutative: $$a*_2b=b*_2a, \forall a,b \in F$$ 
	\end{enumerate}
	
	\item For every non-zero element $a \in F$, we must get a unique element $a'\in F$ such that, $a*_2a'=a'*_2a=e_2$.
\end{enumerate}
\begin{ex} Check the following (Whether Field or not)
\begin{enumerate} 
	\item The set of integers with usual addition and multiplication
	\\ Solution:- 	\\Closure Property: $$a+b \in Z$$
	\\Associative Property: $$a+(b+c)=(a+b)+c, \forall a,b,c \in Z$$
	\\Identity Property: $$a+0=0+a=a, \forall a \in Z$$
	\\Inverse Property: $$ \ a + \ (-a\ ) =\ (-a\ ) + a = 0, \forall a \in Z$$
	\\Commutative Property: $$a+b=b+a,\forall a,b \in Z$$
	\\So, the set of integers with usual addition is an abelian group.\\
	\\For multiplication:
	\begin{itemize}
	    \item Closure: $$\forall a, b \in Z \Rightarrow a*b \in Z$$
        \item Associative: $$a*(b*c)=(a*b)*c, \forall a,b,c \in Z$$
	    \item Identity: For any $a \in Z$, $1$ is an identity element of $Z$ such that, $$a*1=1*a=a$$ .
		\item Distributive of $+$ over $*$:\\
	        $a*(b+c)=(a*b)+ (a*c)$  $ \forall a,b,c \in Z $ \\
	        $(a+b)*c=(a*c)+(b*c)$   $ \forall a,b,c \in Z $ \\
	   \item Commutative: $$a*b=b*a  \forall a,b \in Z$$ \\
	   \item Multiplicative Inverse: $$a*a^{-1}= a^{-1} * a = 1 , \forall a^{-1} \notin Z$$\\
	    \\Therefore the set of integers with usual addition and multiplication is not a field.
	 \end{itemize}
	\item The set of real numbers with usual addition and multiplication
	\\ Solution:- 	\\Closure Property: $$a+b \in R$$
	\\Associative Property: $$a+(b+c)=(a+b)+c, \forall a,b,c \in R$$
	\\Identity Property: $$a+0=0+a=a, \forall a \in R$$
	\\Inverse Property: $$ \ a + \ (-a\ ) =\ (-a\ ) + a = 0, \forall a \in R$$
	\\Commutative Property: $$a+b=b+a,\forall a,b \in R$$
	\\So, the set of real numbers with usual addition is an abelian group.\\
	\\For multiplication:
	\begin{itemize}
	    \item Closure: $$\forall a, b \in R \Rightarrow a*b \in R$$
        \item Associative: $$a*(b*c)=(a*b)*c, \forall a,b,c \in R$$
	    \item Identity: For any $a \in R$, $1$ is an identity element of $R$ such that, $$a*1=1*a=a$$ .
		\item Distributive of $+$ over $*$:\\
	        $a*(b+c)=(a*b)+ (a*c)$  $ \forall a,b,c \in R $ \\
	        $(a+b)*c=(a*c)+(b*c)$   $ \forall a,b,c \in R $ \\
	   \item Commutative: $$a*b=b*a  \forall a,b \in R$$ \\
	   \item Multiplicative Inverse: $$a*a^{-1}= a^{-1} * a = 1 , \forall a^{-1} \in R$$\\
	    \\Therefore the set of real numbers with usual addition and multiplication is a field.
	 \end{itemize}
	\item The set of rational numbers with usual addition and multiplication
	\\ Solution:- 	\\Closure Property: $$a+b \in Q$$
	\\Associative Property: $$a+(b+c)=(a+b)+c, \forall a,b,c \in Q$$
	\\Identity Property: $$a+0=0+a=a, \forall a \in Q$$
	\\Inverse Property: $$ \ a + \ (-a\ ) =\ (-a\ ) + a = 0, \forall a \in Q$$
	\\Commutative Property: $$a+b=b+a,\forall a,b \in Q$$
	\\So, the set of rational numbers with usual addition is an abelian group.\\
	\\For multiplication:
	\begin{itemize}
	    \item Closure: $$\forall a, b \in Q \Rightarrow a*b \in Q$$
        \item Associative: $$a*(b*c)=(a*b)*c, \forall a,b,c \in Q$$
	    \item Identity: For any $a \in Q$, $1$ is an identity element of $Q$ such that, $$a*1=1*a=a$$ .
		\item Distributive of $+$ over $*$:\\
	        $a*(b+c)=(a*b)+ (a*c)$  $ \forall a,b,c \in Q $ \\
	        $(a+b)*c=(a*c)+(b*c)$   $ \forall a,b,c \in Q $ \\
	   \item Commutative: $$a*b=b*a  \forall a,b \in Q$$ \\
	   \item Multiplicative Inverse: $$a*a^{-1}= a^{-1} * a = 1 , \forall a^{-1} \in Q$$\\
	    \\Therefore the set of rational numbers with usual addition and multiplication is a field.
	 \end{itemize}
	\item The set of Even integers with usual addition and multiplication
		\\ Solution:- 	\\Closure Property: $$a+b \in 2N$$
	\\Associative Property: $$a+(b+c)=(a+b)+c, \forall a,b,c \in 2N$$
	\\Identity Property: $$a+0=0+a=a, \forall a \in 2N$$
	\\Inverse Property: $$ \ a + \ (-a\ ) =\ (-a\ ) + a = 0, \forall a \in 2N$$
	\\Commutative Property: $$a+b=b+a,\forall a,b \in 2N$$
	\\So, the set of even integers with usual addition is an abelian group.\\
	\\For multiplication:
	\begin{itemize}
	    \item Closure: $$\forall a, b \in R \Rightarrow a*b \in 2N$$
        \item Associative: $$a*(b*c)=(a*b)*c, \forall a,b,c \in 2N$$
	    \item Identity: For any $a \in Q$, $1$ is an identity element of $2N$ such that, $$a*1=1*a=a$$ .
		\item Distributive of $+$ over $*$:\\
	        $a*(b+c)=(a*b)+ (a*c)$  $ \forall a,b,c \in 2N $ \\
	        $(a+b)*c=(a*c)+(b*c)$   $ \forall a,b,c \in 2N $ \\
	    \item Commutative: $$a*b=b*a  \forall a,b \in 2N$$ \\
	    \item Multiplicative Inverse: $$a*a^{-1} = a^{-1} * a = 1 , \forall a^{-1} \notin 2N$$\\
	    \\Therefore the set of even integers with usual addition and multiplication is not a field.
    \end{itemize}	
\end{enumerate}
\end{ex}
\includegraphics{images/diagram1.png}
\subsection{Finite Fields of the form GF(p)}
\begin{itemize}
    \item Finite fields play a crucial role in many cryptographic algorithms.
    \item The order of a finite field (number of elements in the field) must be a power of a prime $p^{n} $ , where n is a positive integer.
    \item The finite field of order $p^{n} $ is generally written GF($p^{n} $); GF stands for Galois field.
    \item For a given prime, p, we define the finite field of order p, GF(p), as the set $Z_p$ of integers {0, 1, … , p - 1} together with the arithmetic operations modulo p. 
    \item Any integer in $Z_n$ has a multiplicative inverse if and only if that integer is relatively prime to n.
    \item If n is prime, then all of the nonzero integers in $Z_n$ are relatively prime to n, and therefore there exists a multiplicative inverse for all of the nonzero integers in $Z_n$.
    \item For example:- $Z_5$={0,1,2,3,4}
    \\1 mod 5 = 1; 1 is multiplicative inverse of itself
    \\(2*3) mod 5 = 1; 2 is multiplicative inverse of 3
    \\(3*2) mod 5 = 1; 3 is multiplicative inverse of 2
    \\(4*4)mod 5 = 1; 4 is multiplicative inverse of itself
    \\
    \item If a and b are relatively prime, then b has a multiplicative inverse modulo a. That is, if gcd(a, b) = 1, then b has a multiplicative inverse modulo a.
    \item Galois Field
    $$GF(P)=(Z_p,\oplus,\otimes)$$
    \\for $GF(2^{n})$, we use irreducible polynomial
\end{itemize}
\end{defn}
\section{Polynomial arithmetic}
\subsection{Ordinary Polynomial Arithmetic}
\begin{itemize}
    \item A polynomial of degree n (integer $n \geq 0$) is an expression of the form
    $$ $$ f(x)= $a_n X^{n} + a_{n-1} X^{n-1} + a_{n-2} X^{n-2}+...+a_0$= $\sum_{i=0} ^{n} a_{i} X^{i}$
    \item A zero-degree polynomial is called a constant polynomial and is simply an element of the set of coefficients. An nth-degree polynomial is said to be a monic polynomial if $a_n$=1.
\end{itemize}
\subsection{Polynomial Arithmetic with Coefficients in $Z_p$}
\begin{itemize}
    \item When polynomial arithmetic is performed on polynomials over a field, then division is possible.
    \item Note:- this does not mean that exact division is possible. Within a field, given two elements and , the quotient a/b is also an element of the field. However, given a ring R that is not a field, in general, division will result in both a quotient and a remainder; this is not exact division.
    \includegraphics{images/diagram2.png}
\end{itemize}
\subsection{Irreducible Polynomial}
A polynomial f(x) over a field is called irreducible polynomial if and only if f(x) cannot be expressed as a product of two polynomials both over and of degree lower than that of f(x). An irreducible polynomial is also called as prime polynomial.\\
\includegraphics{images/diagram3.png}
\subsection{Finding the Greatest Common Divisor}
We can extend the analogy between polynomial arithmetic over a field and integer
arithmetic by defining the greatest common divisor as follows.The polynomial
is said to be the greatest common divisor of a(x) and b(x)if the following are true.
\begin{enumerate}
    \item c(x) divides both a(x) and b(x).
    \item Any divisor of a(x) and b(x) is a divisor of c(x).
\end{enumerate}
We can adapt the Euclidean algorithm to compute the greatest common divisor of two polynomials.\\
\includegraphics{images/diagram4.png}
\section{Symmetric Cipher Model:-}
A symmetric encryption scheme has five ingredients
\begin{itemize}
    \item Plaintext: This is the original intelligible message or data that is fed into the algorithm as input.
    \item Encryption algorithm: The encryption algorithm performs various substitutions and transformations on the plaintext.
    \item Secret key: The secret key is also input to the encryption algorithm.The key is a value independent of the plaintext and of the algorithm. The algorithm will produce a different output depending on the specific key being used at the time.The exact substitutions and transformations performed by the algorithm depend on the key.
    \item Ciphertext: This is the scrambled message produced as output. It depends on the plaintext and the secret key. For a given message, two different keys will produce two different ciphertexts. The ciphertext is an apparently random stream of data and, as it stands, is unintelligible.
    \item Decryption algorithm: This is essentially the encryption algorithm run in reverse. It takes the ciphertext and the secret key and produces the original plaintext.
\end{itemize}

There are two requirements for secure use of conventional encryption:
\begin{enumerate}
    \item We need a strong encryption algorithm.
    \item Sender and receiver must have obtained copies of the secret key in a secure manner.
\end{enumerate}
\includegraphics{images/diagram5.png}
In Symmetric Cipher Model, we assume that it is impractical to decrypt a message on the basis of the ciphertext plus knowledge of the encryption/decryption algorithm. In other words,we do not need to keep the algorithm secret; we need to keep only the key secret.
\subsection{Cryptography}
Cryptographic systems are characterized along three independent dimensions:
\begin{enumerate}
    \item The type of operations used for transforming plaintext to ciphertext.
    \item The number of keys used.
    \item The way in which the plaintext is processed.
\end{enumerate}
\subsection{Cryptanalysis and Brute-Force Attack}
A brute-force attack involves trying every possible key until an intelligible translation of the ciphertext into plaintext is obtained.\\
In contrast to attacking via brute force, if the attacker has information about which keys are more likely than others and use such information to try and learn the key, then such attack becomes cryptanalysis.
\section{Substitution Technique:-}
The two basic building blocks of all encryption techniques are substitution and transposition.\\
In substitution technique we substitute plain text with other letters,symbols or numbers.\\
For Example:-\\
Plain Text: H E L L O\\
Key: 3\\
Cipher Text:K H O O R\\
\begin{enumerate}
    \item Monoalphabetic Cipher :A monoalphabetic cipher is any cipher in which the letters of the plain text are mapped to cipher text letters based on a single alphabetic key. Examples of monoalphabetic ciphers would include the Caesar-shift cipher, where each letter is shifted based on a numeric key.
    \item Polyalphabetic Cipher : A polyalphabetic cipher is any cipher based on substitution, using multiple substitution alphabets.
\end{enumerate}
\begin{itemize}
    \item Monoalphabetic Cipher : Caesar Cipher, Affine Cipher
    \item Polyalphabetic Cipher : Vigenère Cipher, Vernam Cipher
\end{itemize}
\subsection{Caesar Cipher}
The Caesar Cipher technique is one of the earliest and simplest method of encryption technique. It’s simply a type of substitution cipher, i.e., each letter of a given text is replaced by a letter some fixed number of positions down the alphabet.\\
The encryption can be represented using modular arithmetic by first transforming the letters into numbers, according to the scheme, A = 0, B = 1,…, Z = 25.
\\ Encryption of a letter by a shift n can be described mathematically as.
$$Encrption: E_n(x)=(x+n)mod 26$$
$$Decrption: D_n(x)=(x-n)mod 26$$
\\For Example: if key=3, then:\\
\includegraphics{images/diagram6.png}
\\ If it is known that a given ciphertext is a Caesar cipher, then a brute-force cryptanalysis is easily performed: simply try all the 25 possible keys.\\
Three important characteristics of Caesar Cipher enabled us to use a brute-force cryptanalysis:
\begin{enumerate}
    \item The encryption and decryption algorithms are known.
    \item There are only 25 keys to try.
    \item The language of the plaintext is known and easily recognizable.
\end{enumerate}
\subsection{Affine Cipher:}
The Affine cipher is a type of monoalphabetic substitution cipher, wherein each letter in an alphabet is mapped to its numeric equivalent, encrypted using a simple mathematical function, and converted back to a letter.\\
\\ Encryption: It uses modular arithmetic to transform the integer that each plaintext letter corresponds to into another integer that correspond to a ciphertext letter. The encryption function for a single letter is\\
E ( x ) = ( a x + b ) mod m \\
modulus m: size of the alphabet\\
a and b: key of the cipher.\\
a must be chosen such that a and m are coprime.\\
\\ Decryption: D ( x ) = $a^{-1}$ ( x - b ) mod m
$a^{-1}$ : modular multiplicative inverse of a modulo m. i.e., it satisfies the equation 1 = a.$a^{-1}$ mod m .\\
\includegraphics{images/diagram7.png}
\subsection{Vigenère Cipher}
\begin{itemize}
    \item The table consists of the alphabets written out 26 times in different rows, each alphabet shifted cyclically to the left compared to the previous alphabet, corresponding to the 26 possible Caesar Ciphers.
    \item At different points in the encryption process, the cipher uses a different alphabet from one of the rows.
    \item The alphabet used at each point depends on a repeating keyword.
\end{itemize}
\begin{ex}
Plaintext : GEEKSFORGEEKS\\
Keyword : AYUSH\\
Ciphertext : GCYCZFMLYLEIM\\
\includegraphics{images/diagram8.png}
\\For generating key, the given keyword is repeated in a circular manner until it matches the length of the plain text.\\
The keyword "AYUSH" generates the key "AYUSHAYUSHAYU"\\
The plain text is then encrypted using the process explained below.
\\Encryption: The first letter of the plaintext, G is paired with A, the first letter of the key. So use row G and column A of the Vigenère square, namely G. Similarly, for others. Using Algebracially it can be as:\\
The plaintext(P) and key(K) are added modulo 26.\\
$E_i = (P_i + K_i) mod 26$
\\
Decryption:\\
$D_i = (E_i - K_i + 26) mod 26$
\end{ex}
\subsection{Vernam Cipher:}
In this mechanism we assign a number to each character of the Plain-Text, like (a = 0, b = 1, c = 2, … z = 25).\\
Method to take key:
\\ In Vernam cipher algorithm, we take a key to encrypt the plain text which length should be equal to the length of the plain text.\\
Encryption Algorithm:
\begin{itemize}
    \item Assign a number to each character of the plain-text and the key according to alphabetical order.
    \item Add both the number (Corresponding plain-text character number and Key character number).
    \item Subtract the number from 26 if the added number is greater than 26, if it isn’t then leave it.
\end{itemize}    
\begin{ex}
Plain-Text: RAMSWARUPK\\
Key: RANCHOBABA\\
PT:   R  A  M   S   W   A  R   U   P   K\\
NO:   17 0  12  18  22  0  17  20  15  10\\
KEY:  R   A  N   C  H  O   B  A  B  A\\ 
NO :  17  0  13  2  7  14  1  0  1  0\\
on adding Both:\\
CT-NO: 34  0  25  20  29  14  18  20  16  10 \\
In this case, there are two numbers which are greater than the 26 so we have to subtract 26 from them and after applying the subtraction operation the new Cipher text character numbers are as follow:\\
CT-NO:  8  0  25   20   3   14   18   20   16   10\\
New Cipher-Text is after getting the corresponding character from the number.\\
CIPHER-TEXT: I  A  Z  U  D  O  S  U  Q  K 
\end{ex}
\subsection{Playfair Cipher}
The Playfair Cipher Encryption Algorithm:\\
The Algorithm consistes of 2 steps:\\
Generate the key square(5*5) 
\begin{itemize}
        \item The key square is a 5*5 grid of alphabets that acts as the key for encrypting the plaintext. Each of the 25 alphabets must be unique and one letter of the alphabet (usually J) is omitted from the table (as the table can hold only 25 alphabets). If the plaintext contains J, then it is replaced by I.
        \item The initial alphabets in the key square are the unique alphabets of the key in the order in which they appear followed by the remaining letters of the alphabet in order.
\end{itemize}
Algorithm to encrypt the plain text: The plaintext is split into pairs of two letters \begin{itemize}
    \item If both the letters are in the same column: Take the letter below each one (going back to the top if at the bottom).
    \item If both the letters are in the same row: Take the letter to the right of each one (going back to the leftmost if at the rightmost position). 
    \item If neither of the above rules is true: Form a rectangle with the two letters and take the letters on the horizontal opposite corner of the rectangle. 
\end{itemize}
\begin{ex}
Using Key "monarchy" \\
\includegraphics{images/diagram9.png}
\\ Plain Text: "instrumentsz"\\
Encrypted Text: gatlmzclrqtx\\
Encryption: \\in:
  i $ ->$ g\\
  n $ ->$ a\\st:
  s $ ->$ t\\
  t $ ->$ l\\ru:
  r $ ->$ m\\
  u $ ->$ z\\me:
  m $ ->$ c\\
  e $ ->$ l\\nt:
  n $ ->$ r\\
  t $ ->$ q\\sz:
  s $ ->$ t\\
  z $ ->$ x\\
\end{ex}
\subsection{Hill Cipher}
Hill cipher is a polygraphic substitution cipher based on linear algebra.Each letter is represented by a number modulo 26. To encrypt a message, each block of n letters (considered as an n-component vector) is multiplied by an invertible n * n matrix, against modulus 26. To decrypt the message, each block is multiplied by the inverse of the matrix used for encryption.\\
The matrix used for encryption is the cipher key, and it should be chosen randomly from the set of invertible n * n matrices.
\begin{ex}
We have to encrypt the message ‘ACT’ (n=3).\\
The key is ‘GYBNQKURP’ 
\\
\includegraphics{images/diagram10.png}
\end{ex}
\subsection{Pigpen Cipher}
The Pigpen Cipher is another example of a substitution cipher, but rather than replacing each letter with another letter, the letters are replaced by symbols.
\\Encryption:\\
The encryption process is fairly straightforward, replacing each occurence of a letter with the designated symbol. The symbols are assigned to the letters using the key shown below, where the letter shown is replaced by the part of the image in which it is located.\\
\includegraphics{images/diagram11.png}
\\ Decryption\\
The decryption process is just the reverse of the encryption process. Using the same key (the grid above), you locate the image depicted in the ciphertext, and replace it with the letter given by that part of the grid.
\subsection{Enigma Machine}
The Enigma machine is an encryption device developed and used in the early- to mid-20th century to protect commercial, diplomatic and military communication. It was employed extensively by Nazi Germany during World War II, in all branches of the German military.\\
Enigma has an electromechanical rotor mechanism that scrambles the 26 letters of the alphabet. In typical use, one person enters text on the Enigma's keyboard and another person writes down which of 26 lights above the keyboard lights up at each key press. If plain text is entered, the lit-up letters are the encoded ciphertext. Entering ciphertext transforms it back into readable plaintext. The rotor mechanism changes the electrical connections between the keys and the lights with each keypress. The security of the system depends on a set of machine settings that were generally changed daily during the war, based on secret key lists distributed in advance, and on other settings that were changed for each message. The receiving station has to know and use the exact settings employed by the transmitting station to successfully decrypt a message.
\section{Transposition Technique}
A very different kind of mapping is achieved by performing
some sort of permutation on the plaintext letters. This technique is referred to as a
transposition cipher.
\subsection{Rail Fence Cipher}
The rail fence cipher (also called a zigzag cipher) is a form of transposition cipher.
It derives its name from the way in which it is encoded.\\
Encryption\\
Input : "attack at once"\\
Key = 2 \\
Output : atc toctaka ne \\
Decryption\\
Input : "atc toctaka ne"\\
Key = 2\\
Output : attack at once\\
\section{Steganography}
Steganography is the technique of hiding secret data within an ordinary, non-secret, file or message in order to avoid detection; the secret data is then extracted at its destination. The use of steganography can be combined with encryption as an extra step for hiding or protecting data.
\section{Fundamental Theorem of Arithmetic:}
The fundamental theorem of arithmetic states that every positive integer (except the number 1) can be represented in exactly one way apart from rearrangement as a product of one or more primes.\\
This theorem is also called the unique factorization theorem. The fundamental theorem of arithmetic is a corollary of the first of Euclid's theorems.\\
\includegraphics{images/diagram12.png}
\section{Euler’s Totient Function:}
Euler's totient function, also known as phi-function $\phi(n)$, counts the number of
integers between 1 and n inclusive, which are co-prime to n. Two numbers are co-prime if
their greatest common divisor equals 1.\\
Here are values of $\phi(n)$ for the first few positive integers:\\
\includegraphics{images/diagram13.png}
\subsection{Properties:}
The following properties of Euler totient function are sufficient to calculate it for any number:
\begin{itemize}
    \item If p is a prime number, then GCD(p,q) = 1, $\forall 1 \leq q < p $.
    \\Therefore we have: $\phi(p)=p-1$
    \item If a and b are relatively prime, then: 
    \\ $\phi(pq)= \phi(p) * \phi(q) = (p-1)(q-1) $
    \item If p is a prime number and $k\geq 1$, then there are exactly $p^{k}$/p numbers between 1 and $p^{k}$ that are divisible by p. Which gives us: 
    $$ \phi(p^{k})=p^{k}-p^{k-1}$$
\end{itemize}
\section{Fermat's Little theorem:}
Fermat's little theorem is a fundamental theorem in elementary number theory, which helps compute powers of integers modulo prime numbers.
\\ Let p be any prime number and a be any integer. Then $a^{p}$-p is always divisible by p. \\
Modular arithmetic notation: $a^{p}$ $\equiv$ a mod p \\
$a^{p-1}$ $\equiv$ 1 mod p \\
For example: a = 7, p = 19 \\
$7^{2}$ = 49 = 11(mod 19)\\
$7^{4}$ = 121 = 7(mod 19)\\
$7^{8}$ = 49 = 11(mod 19)\\
$7^{16}$ = 121 = 7(mod 19)\\
$a^{p-1}$ = $7^{18}$ = $7^{16}$ * $7^{2}$ = 7*11 = 1(mod 19)
\section{System of Linear Congruences}
1. Using solution sets.\\
2. Using equations. \\
3. Chinese Remainder Theorem (C.R.T)
\subsection{Using solution sets:}
x = 1 mod 3 \\
x = 3 mod 5 \\
A1 = {..., -2, 1, 4, 7, 10,...} \\
A2 = {..., -2, 3, 8, 13, 18,...} \\
solution set for x = A1 $\cap$ A2
\subsection{Using Equation:}
x $\equiv$ 3 mod 5   \\
x $\equiv$ 1 mod 3   \\
x=5q1 + 3   ..eq(1)   \\
x=3q2 + 1   ..eq(2)   \\
5q1 = 3q2  - 2   \\
5q1 mod 3 = 1 mod 3   \\
q1 5*2 mod 3 = 2 mod 3   \\
q1 = 3k +2   \\
using eq(1):   \\
x = 5q1 + 3   \\
x = 5(3k + 2) +3   \\
x = 15k + 13
\subsection{Chinese Remainder Theorem (C.R.T)}
Chinese remainder theorem gives us an algorithm for solving a system of linear congruences with one unknown. \\
Proof:\\
$M=m_1*m_2*m3*...*m_r$  and  $M_j=M/m_j$ , for j = 1 to r \\
Let $y_i$be an inverse of $M_j$ modulo $m_j$ \\
Then , x = $\sum_{1}^{r} a_j M_j y_j$ Mod M \\
For Example: \\
x $\equiv$ a1 mod n1 \\
x $\equiv$ a2 mod n2 \\
x $\equiv$ a3 mod n3 \\
M = a1*a2*a3 \\
m1 = M/a1 \\
m2 = M/a2 \\
m3 = M/a3 \\
y1 = ($m1^{-1}$) Mod a1 \\
y2 = ($m2^{-1}$) Mod a2 \\
y3 = ($m3^{-1}$) Mod a3 \\
x = $\sum_{1}^{3} a_i M_i y_i$ Mod M 
\section{RSA Algorithm:}
RSA algorithm is asymmetric cryptography algorithm. Asymmetric actually means that it works on two different keys i.e. Public Key and Private Key. As the name describes that the Public Key is given to everyone and Private key is kept private.\\
The idea of RSA is based on the fact that it is difficult to factorize a large integer. The public key consists of two numbers where one number is multiplication of two large prime numbers. And private key is also derived from the same two prime numbers. So if somebody can factorize the large number, the private key is compromised. Therefore encryption strength totally lies on the key size and if we double or triple the key size, the strength of encryption increases exponentially. RSA keys can be typically 1024 or 2048 bits long, but experts believe that 1024 bit keys could be broken in the near future. But till now it seems to be an infeasible task.
\subsection{Algorithm:}
\begin{itemize}
    \item Step 1: Choose large p,q
    \item Step 2: n = p.q
    \item Step 3: Choose e: $1<e<\phi(n)$
    \item Step 4: Calculate d: d = $e^{-1}$ Mod $\phi(n)$
    \item Step 5: Public Key (e,n)\\
    Priavte Key (d,p,q) - secret key \\
    $\phi(n)$ = (p-1)*(q-1)
    \item Step 6: C = $M^{e}$ Mod n
    \\ M = $C^{d}$ Mod n
\end{itemize}
\section{Quadratic Residue Mod p}
If there is an integer $0<x<p$ such that\\
$x^{2}=q$(mod p)\\
the congruence has a solution, then q is said to be a quadratic residue (mod p).Note that the trivial case q=0 is generally excluded from lists of quadratic residues so that the number of quadratic residues (mod n) is taken to be one less than the number of squares (mod n). However, other sources include 0 as a quadratic residue. \\
If the congruence does not have a solution, then q is said to be a quadratic non-residue (mod p).\\
In practice, it suffices to restrict the range to $0<x<=\lfloor p/2 \rfloor$, where $\lfloor x \rfloor$ is the floor function, because of the symmetry $(p-x)^2=x^2$ (mod p). 
\\ For Example: $Z_7$={0,1,2,3,4,5,6}
\\ $1^{2}$ Mod 7 = 1
\\ $2^{2}$ Mod 7 = 4
\\ $3^{2}$ Mod 7 = 2
\\ $4^{2}$ Mod 7 = 2
\\ $5^{2}$ Mod 7 = 4
\\ $6^{2}$ Mod 7 = 1
\\ Q R Mod P = {1,2,4} \\
x1 = r then x2 = -r + p \\
For Quadratic Non-Residue Mod p \\
No. of $QNR_p$ Mod p $\Rightarrow$ (p-1)/2 \\
No. of $QR_p$ Mod p $\Rightarrow$ (p-1)/2 
\section{Euler's Criterion :}
Let p be a prime number, then for any x $\in$ $Z_p^{*}$ , x $\in$ QR Mod p,iff
\\ $a^{(p-1)/2}$ = 1 Mod p
\\For Example: Let p = 29 and a = 13
\\ $13^{(29-1)/2}$ $\equiv$ $13^{14}$ \\
13 Mod 29 = 13  \\
$13^{2}$ Mod 29 = 24  \\
$13^{4}$ Mod 29 = $24^{2}$ Mod 29 = 25  \\
$13^{8}$ Mod 29 = $25^{2}$ Mod 29 = 16  \\
16*25*24 Mod 29 $\equiv$ 9600 Mod 29 $\equiv$ 1
\section{Legendre Symbol :}
Legendre Symbol is defined to be equal to $\pm$1 depending on whether 'a' is a quadratic residue modulo p. \\
\includegraphics{images/diagram14.png}
\\ For Example:
\\ (169/13) = 0 \\
(2/13) = -1 \\
$2^{(13-1)/2}$ Mod 13 \\
$2^{6}$ Mod 13 \\
64 Mod 13
\section{Jacobi Symbol :}
The Jacobi symbol is a generalization of the Legendre symbol, which can be used to simplify computations involving quadratic residues. It shares many of the properties of the Legendre symbol, and can be used to state and prove an extended version of the law of quadratic reciprocity. \\
Jacobi Symbol (JS) $\Rightarrow$ Legendre Symbol (LS) \\
Legendre Symbol (LS) doesn't implies Jacobi Symbol \\
\includegraphics{images/diagram15.png}\\
$P_k$ is distinct prime nos.\\
(a/n) =  $\sum_{1}^{k} (a/P_i)^{ei}$

Primality test can be classified into
\begin{enumerate}
    \item Deterministic ( prime with 100 percent true)
    \item Probablistic (Composite then 100 percent true, prime then may or may not be true)
\end{enumerate}
To check prime (Primality Test)
\begin{enumerate}
    \item Sieve of Eratosthenes
    \item Square Root Method
    \item Mersenne Prime
    \item Fermat's Prime
    \item Solovay strassen Algorithm
    \item Miller-Rabin Primality Test
    
\end{enumerate}
\section{Sieve of Eratosthenes:}
The sieve of Eratosthenes is one of the most efficient ways to find all primes smaller than n.
\\Steps to find all the prime numbers less than or equal to a given integer n by the Eratosthene’s method:
\begin{enumerate}
    \item Create a list of consecutive integers from 2 to n: (2, 3, 4, …, n).
    \item Initially, let p equal 2, the first prime number.
    \item Starting from $p^{2}$, count up in increments of p and mark each of these numbers greater than or equal to $p^{2}$ itself in the list. These numbers will be p(p+1), p(p+2), p(p+3), etc..
    \item Find the first number greater than p in the list that is not marked. If there was no such number, stop. Otherwise, let p now equal this number (which is the next prime), and repeat from step 3.
\end{enumerate}

\section{Fermat Primality Test:}
\begin{enumerate}
    \item Add odd integer, n $>$ 0
    \item Choose random 'a', $\exists$ a $<$ n
    \item If the GCD(a,n) $\neq$ 1
    \\ Then n - not prime.
    \item If $a^{n-1}$ MOD n $\neq$ 1
    \\ Then n - composite.
    \item If $a^{n-1}$ MOD n $=$ 1
    \\ Then n - may be prime.
\end{enumerate}
\begin{ex}
a = 4, n = 15
\\ $a^{n-1}$ = 1 MOD 15
\\ $4^{1}$ = 4 MOD 15
\\ $4^{2}$ = 1 MOD 15
\\ $4^{4}$ = 1 MOD 15
\\ $4^{8}$ = 1 MOD 15
\\ $4^{14}$ = $4^{2+4+8}$ = 1 MOD 15
\end{ex}
\section{Square Root Test:}
Using Q.R MOD p
\\ $x^{2}$ = a MOD p
\\ $\sqrt a$ = +x, -x+p
\begin{ex}
3 in $Z_7$
\\ $1^{2}$ MOD 7 = 1
\\ $2^{2}$ MOD 7 = 4
\\ $3^{2}$ MOD 7 = 2
\\ $4^{2}$ MOD 7 = 2
\\ $5^{2}$ MOD 7 = 4
\\ $6^{2}$ MOD 7 = 1

\end{ex}
Square root of 1 MOD n is $\pm$ 1.
\\ Square root of 1 MOD n is $\pm$ 1 , $\pm$ r, n is integer
\\ P-primes $\longrightarrow$ 6n $\pm$ 1 For numbers greater than 3.
\section{Solovay strassen Algorithm:}
\begin{enumerate}
    \item Start with an odd positive integer n.
    \item Choose a random integer 'a' such that 1 < a < n-1.
    \item If GCD(a,n) $\pm$ 1, then n composite.
    \item Find x = (a/n)
    \item Calculate j = $a^{(n-1)/2}$ MOD n
    \item If j $\pm$ (a/n), then n is not prime.
    \item If j = (a/n), then n may be prime.
\end{enumerate}
\begin{ex}
n=15, a= 4
\begin{enumerate}
    \item n = 15
    \item a = 4
    \item GCD(4,15) = 1
    \item x = 4/15
    \item j = $4^{14/2}$ MOD 15
    \\j = $4^{7}$ MOD 15
    \\j = 4
    \item j $\neq$ x. So, not prime
\end{enumerate}
\end{ex}
\section{Miller Rabin Primality Test}
Choose an odd integer n.
\begin{enumerate}
    \item Find integer k,q with k > 0, q $\in$ odd such that n-1 = $2^{k}$q
    \item Select random integer a, such that a $\in$ (1,n-1)
    \item If $a^{q}$Mod n = 1, then return "inconclusive".
    \item Else for j=1 to k-1 do
    \\if $a^{{2}^{j}q}$ MOD n = n-1 then return "Conclusive"
    \\else return Composite
\end{enumerate}
\begin{ex}
n=29
\\1. n-1=28=$2^{2}.7$
\\k=2,q=7
\\2. Let a = 10, (1<a<28)
\\3. $a^{q}$Mod n: $10^{7}$Mod 29=17
\\n-may be prime
\end{ex}
\section{Carmicheal Number}
pseudo prime
\\ It is nothing a composite number 'n' that satisfies:
\\n $\rightarrow$ $p^{n-1}$ $\equiv$ b MOD n
\\ $\forall$ b,GCD(b,n) = 1
\\n-1 = $\pi$ $p^{e_i}_i$ = (n-1)/(p-1) 
\section{Factorization Algorithm}
\subsection{Pollard P-1 Factoring Algorithm}
\begin{lstlisting}
    Given a number n.
    Initialize a = 2, i = 2
    Until a factor is returned do
    a <- (a^i) mod n
    d <- GCD(a-1, n)
    if 1 < d < n then
        return d
    else
        i <- i+1
    Other factor, d' <- n/d
    If d' is not prime
        n <- d'
        goto 1
    else
        d and d' are two prime factors. 
\end{lstlisting}
In this algorithm, the power of ‘a’ is continuously raised until a factor, ‘d’, of n is obtained. Once d is obtained, another factor, ‘d”, is n/d. If d’ is not prime, the same task is repeated for d’.
\subsection{Pollard Rho factorizing Algorithm:}
\includegraphics{images/diagram16.png}
\end{document}
