\documentclass{article}
\usepackage[utf8]{inputenc}
\usepackage{graphicx}
\graphicspath{ {./images/} }
\newcommand\tab[1][3cm]{\hspace*{#1}}
\newtheorem{defn}{Definition} [section]
\newtheorem{ex}{Example}[section]
\title{Chapter-1 "ALGEBRA"}
\author{Shubham Kumar\thanks{Mtech-1st yr: Information Security, Atal Bihari Vajpayee Indian Institute of Information Technology and Management, Gwalior, India-474015} }
\date{October 2020}

\begin{document}

\maketitle

\section{Set Theory:-}
\begin{itemize}
    \item Universal Set or Universe of discourse: Collection of Objects.
	\item Set: Collection of well-defined objects. Here the term well defined refers, the definition of a set is not person dependent.
	\item It can be defined using Characteristic function. This tells us that the entire (any) Mathematical structure is built on the binary logic.
	\item Cartesian product of two sets $A$ and $B$ is nothing but the set of all possible combinations of elements of $A$ and $B$
	\item Function: Every element of domain is associated with some other (unique) element of the co-domain
	\item Binary Operator: $*$ is said to be a binary operator and is defined as $*:G \times G \rightarrow G $
	\item Arithmetic Operation: Addition, Subtraction, Multiplication, Division (all are binary operators on $\Re$)
\end{itemize}
\section{Group Theory:-}
\begin{defn}
A set $G$ with a Binary operation $*$ defined on $G$ is said to be a {\textbf{Group}}, if it satisfies the following four axioms,
\begin{itemize}
	\item Closure: $$\forall a, b \in G \Rightarrow a*b \in G$$
	
	\item Associative: $$a*(b*c)=(a*b)*c, \forall a,b,c \in G$$
	
	\item Identity: For every $a \in G$, there exists a unique element $e$ such that, $a*e=e*a=a$, then 'e' is called as identity element
	\item Inverse: For every non-zero element $a$ of $G$, there exists a unique non-zero $a' \in G$ such that, $$a*a'=a'*a=e$$
	\item Commutative: $$a*b=b*a, \forall a,b \in G$$\\
	If a Group satisfies the Commutative property, then it is called as an Abelian Group.
\end{itemize}
\subsection{Classification of a Algebraic Structure}
\begin{itemize}
    \item Groupoid: Closure
    \item SemiGroup: Closure + Associative
    \item Monoid: Closure + Associative + Identity
    \item Group: Closure + Associative + Identity + Inverse
    \item Abelian Group: If a Group satisfies the Commutative property.
\end{itemize}
\subsection{Properties of Group}
\begin{itemize}
    \item If a group has a finite number of elements, it is referred to as a finite group, and the order of the group is equal to the number of elements in the group. Otherwise, the group is an infinite group.
    \item Identity Element($e$) is unique for a Group.
    \item Left Identity Should be same as Right Identity : $$a*e=e*a$$
    \item $a^{-1}$ is unique for a given element a.
    \item $(a*b)^{-1}=b^{-1}*a^{-1}$
    \item If G is Abelian Group than: $$(a*b)^{-1}=b^{-1}*a^{-1}$$ $$(a*b)^{-1}=a^{-1}*b^{-1}$$
\end{itemize}
\subsubsection{Cyclic Group}
\begin{itemize}
    \item A group G is cyclic if every element of G is a power $a^{k}$ (k is an integer) of a fixed element $a \in G$. The element a is said to generate the group G, or to be a generator of G.We denote the cyclic group of order n by $Z_n$.
    \item A cyclic group is always an abelian group, and may be finite or infinite but and abelian group need not be cyclic group.
    \item A non-abelian group will always be non-cyclic.
    
\end{itemize}
\begin{ex} Check the following (Whether Cyclic or not)
\\For $n \geq 1$ $({Z}_n,\oplus)$ is cyclic
\\Solution: $Z_n$ is an abelian group
\\ For $\oplus$, $Z_n$ generates number $a_i$ where $a_i < n$
\\ Then we can find a subgroup of $Z_n$ which generates a repetative group. So, we can say that $Z_n$ is cyclic group.
\end{ex}
\begin{ex}Check the following (whether Group or not)
\begin{enumerate}
	\item The set of integers with usual addition 
	\\ Solution:- Let $a,b,c \in Z$, then 
	\\Closure Property: $$a+b \in Z$$
	\\Associative Property: $$a+(b+c)=(a+b)+c, \forall a,b,c \in Z$$
	\\Identity Property: $$a+0=0+a=a, \forall a \in Z$$
	\\Inverse Property: $$ \ a + \ (-a\ ) =\ (-a\ ) + a = 0, \forall a \in Z$$
	\\Commutative Property: $$a+b=b+a,\forall a,b \in Z$$
	\\So, the set of integers with usual addition is not only group but it's an abelian group.\\
	\item The set of integers with usual multiplication
	\\ Solution:- Let $a,b,c \in Z$, then 
	\\Closure Property: $$a*b \in Z$$
	\\Associative Property: $$a*(b*c)=(a*b)*c, \forall a,b,c \in Z$$
	\\Identity Property: $$a*1=1*a=a, \forall a \in Z$$
	\\Inverse Property: $$a*a^{-1}= a^{-1} * a = 1 , \forall a \in Z$$
	\\It doesn't holds because except for 1, inverse of every number is not integer it's in fraction.
	\\Therefore, the set of integers under multiplication is not a group.\\
	\item The set of real numbers with usual addition
	\\Solution:- Let $a,b,c \in R$, then 
	\\Closure Property: $$a+b \in R$$
	\\Associative Property: $$a+(b+c)=(a+b)+c, \forall a,b,c \in R$$
	\\Identity Property: $$a+0=0+a=a, \forall a \in R$$
	\\Inverse Property: $$ \ a + \ (-a\ ) =\ (-a\ ) + a = 0 , \forall a \in R$$
	\\Commutative Property: $$a+b=b+a,\forall a,b \in R$$
	\\So, the set of real numbers with usual addition is not only group but it's an abelian group.\\
	\item The set of real numbers with usual multiplication
	\\Solution:- Let $a,b,c \in R$, then 
	\\Closure Property: $$a*b \in R$$
	\\Associative Property: $$a*(b*c)=(a*b)*c, \forall a,b,c \in R$$
	\\Identity Property: $$a*1=1*a=a, \forall a \in R$$
	\\Inverse Property: $$a*a^{-1}= a^{-1} * a = 1 , \forall a \in R$$
	\\Commutative Property: $$a+b=b+a,\forall a,b \in R$$
	\\Therefore, the set of real numbers under multiplication is not only group but it's an abelian group.\\
	\item The set of natural numbers with usual addition
	\\Solution:- Let $a,b,c \in N$, then 
	\\Closure Property: $$a+b \in N$$
	\\Associative Property: $$a+(b+c)=(a+b)+c, \forall a,b,c \in N$$
	\\Identity Property: $$a+0=0+a=a, \forall a \in N$$
	\\Inverse Property: $$ \ a + \ (-a\ ) =\ (-a\ ) + a = 0 , \forall a \in R, but except for a = 0, -a \notin N $$
	\\So, The set of natural numbers under addition is not a group beause it does not have the inverse property .\\
	\item $A=\{0,1,2,3\}, a*b=a+b-ab$
	\\Closure Property: For a=3 and b=3
	\\ $$ a*b=a+b-ab = 3+3-(3*3) $$
    \tab	$a*b= 6-9 = -3$ and $-3 \notin A$ 
    \\So,it is not a group.
\end{enumerate}
\end{ex}
\end{defn}
\section{Rings:-}
\begin{defn} A set $R$ with two Binary operations $*_1$ and $*_2$ (denoted by \textbf{$(R, *_1,*_2)$}) is said to be a \emph{\textbf{Ring}}, if
It satisfies the following axioms,\\
\begin{itemize}
    \item $(R,*_1)$ is an abelian group
	\item Closure: $$\forall a, b \in R \Rightarrow a*_2b \in R$$
	
	\item Associative: $$a*_2(b*_2c)=(a*_2b)*_2c, \forall a,b,c \in R$$
	
	\item Identity: For any $a \in R$, there exists a unique $e_2 \in R$ such that, $$a*_2e_2=e_2*_2a=a$$, then '$e_2$' is called as an identity element of $R$ w.r.t. $*_2$.
		\item Distributive of $*_1$ over $*_2$:\\
	$a*_2(b*_1c)=(a*_2b)*_1 (a*_2c)$\\
	$(a*_1b)*_2c=(a*_1c)*_2 (b*_1c)$\\
	\\ When it satisfies the above mention properties, then it is called ring.
	\item Commutative: $$a*_2b=b*_2a, \forall a,b \in G$$ When a Ring satisfies commutative property w.r.t. $*_2$, then we call $(R,*_1,*_2)$ as a \textbf{\emph{commutative ring with identity}}
\subsection{Integral Domain}
Commutative ring that obeys the following axioms
\begin{itemize}
    \item Multiplicative identity: There is an element 1 in R such that a1=1a=a, $ \forall a \in R $
    \item Non Zero divisors: If a, b in R and ab = 0, then either a = 0 or b=0.
\end{itemize}
\end{itemize}
\begin{ex} Check the following (Whether Ring or not)
\begin{enumerate} 
	\item The set of integers with usual addition and multiplication
	\\ Solution:- Set of integers with usual addition is an abelian group as shown above in example (2.1-1)
	\\For multiplication:
	\begin{itemize}
	    \item Closure: $$\forall a, b \in Z \Rightarrow a*b \in Z$$
        \item Associative: $$a*(b*c)=(a*b)*c, \forall a,b,c \in Z$$
	    \item Identity: For any $a \in Z$, there exists a unique $e_2 \in R$ such that, $$a*1=1*a=a$$, then '$1$' is an identity element of $Z$ w.r.t. $*$.
		\item Distributive of $+$ over $*$:\\
	        $a*(b+c)=(a*b)+ (a*c)$  $ \forall a,b,c \in Z $ \\
	        $(a+b)*c=(a*c)+(b*c)$   $ \forall a,b,c \in Z $ \\
	\\Therefore the set of integers with usual addition and multiplication is a ring.
	\end{itemize}
	\item The set of real numbers with usual addition and multiplication
	\\ Solution:- Set of real numbers with usual addition is an abelian group as shown above in example (2.1-3)
	\\For multiplication:
	\begin{itemize}
	    \item Closure: $$\forall a, b \in R \Rightarrow a*b \in R$$
        \item Associative: $$a*(b*c)=(a*b)*c, \forall a,b,c \in R$$
	    \item Identity: For any $a \in R$, $1$ is an identity element of $R$ such that, $$a*1=1*a=a$$ .
		\item Distributive of $+$ over $*$:\\
	        $a*(b+c)=(a*b)+ (a*c)$  $ \forall a,b,c \in R $ \\
	        $(a+b)*c=(a*c)+(b*c)$   $ \forall a,b,c \in R $ \\
	\\Therefore the set of real numbers with usual addition and multiplication is a ring.
	\end{itemize}
	\item The set of rational numbers with usual addition and multiplication
	\\ Solution:- Let $a,b,c \in Q$, then 
	\\Closure Property: $$a+b \in Q$$
	\\Associative Property: $$a+(b+c)=(a+b)+c, \forall a,b,c \in Q$$
	\\Identity Property: $$a+0=0+a=a, \forall a \in Q$$
	\\Inverse Property: $$ \ a + \ (-a\ ) =\ (-a\ ) + a = 0, \forall a \in Q$$
	\\Commutative Property: $$a+b=b+a,\forall a,b \in Q$$
	\\So, the set of rational numbers with usual addition is an abelian group.\\
	\\For multiplication:
	\begin{itemize}
	    \item Closure: $$\forall a, b \in R \Rightarrow a*b \in Q$$
        \item Associative: $$a*(b*c)=(a*b)*c, \forall a,b,c \in Q$$
	    \item Identity: For any $a \in Q$, $1$ is an identity element of $Q$ such that, $$a*1=1*a=a$$ .
		\item Distributive of $+$ over $*$:\\
	        $a*(b+c)=(a*b)+ (a*c)$  $ \forall a,b,c \in Q $ \\
	        $(a+b)*c=(a*c)+(b*c)$   $ \forall a,b,c \in Q $ \\
	    \\Therefore the set of rational numbers with usual addition and multiplication is a ring.
	\end{itemize}
	\item The set of Even integers(2N) with usual addition and multiplication
	\\ Solution:- 	\\Closure Property: $$a+b \in 2N$$
	\\Associative Property: $$a+(b+c)=(a+b)+c, \forall a,b,c \in 2N$$
	\\Identity Property: $$a+0=0+a=a, \forall a \in 2N$$
	\\Inverse Property: $$ \ a + \ (-a\ ) =\ (-a\ ) + a = 0, \forall a \in 2N$$
	\\Commutative Property: $$a+b=b+a,\forall a,b \in 2N$$
	\\So, the set of even integers with usual addition is an abelian group.\\
	\\For multiplication:
	\begin{itemize}
	    \item Closure: $$\forall a, b \in R \Rightarrow a*b \in 2N$$
        \item Associative: $$a*(b*c)=(a*b)*c, \forall a,b,c \in 2N$$
	    \item Identity: For any $a \in Q$, $1$ is an identity element of $2N$ such that, $$a*1=1*a=a$$ .
		\item Distributive of $+$ over $*$:\\
	        $a*(b+c)=(a*b)+ (a*c)$  $ \forall a,b,c \in 2N $ \\
	        $(a+b)*c=(a*c)+(b*c)$   $ \forall a,b,c \in 2N $ \\
	    \\Therefore the set of even integers with usual addition and multiplication is a ring.
    \end{itemize}	
\end{enumerate}
\end{ex}
\end{defn}
\section{Fields:-}
\begin{defn} A set $F$ with two Binary operations $*_1,*_2$ (denoted by $(F,*_1,*_2)$) is said to be a field, if it satisfies the following axioms,
\begin{enumerate}
	\item $(F,*_1,*_2)$ is a commutative ring with identity 
	\begin{enumerate}
		\item $(F,*_1)$ is an abelian group
	\item Closure: $$\forall a, b \in F \Rightarrow a*_2b \in F$$
	
	\item Associative: $$a*_2(b*_2c)=(a*_2b)*_2c, \forall a,b,c \in F$$
	
	\item Identity: For any $a \in F$, there exists a unique $e_2 \in F$ such that, $$a*_2e_2=e_2*_2a=a$$, then '$e_2$' is called as an identity element of $F$ w.r.t. $*_2$.
		\item Distributive of $*_1$ over $*_2$:\\
	$a*_2(b*_1c)=(a*_2b)*_1 (a*_2c)$\\
	$(a*_1b)*_2c=(a*_1c)*_2 (b*_1c)$
	\item Commutative: $$a*_2b=b*_2a, \forall a,b \in F$$ 
	\end{enumerate}
	
	\item For every non-zero element $a \in F$, we must get a unique element $a'\in F$ such that, $a*_2a'=a'*_2a=e_2$.
\end{enumerate}
\begin{ex} Check the following (Whether Field or not)
\begin{enumerate} 
	\item The set of integers with usual addition and multiplication
	\\ Solution:- 	\\Closure Property: $$a+b \in Z$$
	\\Associative Property: $$a+(b+c)=(a+b)+c, \forall a,b,c \in Z$$
	\\Identity Property: $$a+0=0+a=a, \forall a \in Z$$
	\\Inverse Property: $$ \ a + \ (-a\ ) =\ (-a\ ) + a = 0, \forall a \in Z$$
	\\Commutative Property: $$a+b=b+a,\forall a,b \in Z$$
	\\So, the set of integers with usual addition is an abelian group.\\
	\\For multiplication:
	\begin{itemize}
	    \item Closure: $$\forall a, b \in Z \Rightarrow a*b \in Z$$
        \item Associative: $$a*(b*c)=(a*b)*c, \forall a,b,c \in Z$$
	    \item Identity: For any $a \in Z$, $1$ is an identity element of $Z$ such that, $$a*1=1*a=a$$ .
		\item Distributive of $+$ over $*$:\\
	        $a*(b+c)=(a*b)+ (a*c)$  $ \forall a,b,c \in Z $ \\
	        $(a+b)*c=(a*c)+(b*c)$   $ \forall a,b,c \in Z $ \\
	   \item Commutative: $$a*b=b*a  \forall a,b \in Z$$ \\
	   \item Multiplicative Inverse: $$a*a^{-1}= a^{-1} * a = 1 , \forall a^{-1} \notin Z$$\\
	    \\Therefore the set of integers with usual addition and multiplication is not a field.
	 \end{itemize}
	\item The set of real numbers with usual addition and multiplication
	\\ Solution:- 	\\Closure Property: $$a+b \in R$$
	\\Associative Property: $$a+(b+c)=(a+b)+c, \forall a,b,c \in R$$
	\\Identity Property: $$a+0=0+a=a, \forall a \in R$$
	\\Inverse Property: $$ \ a + \ (-a\ ) =\ (-a\ ) + a = 0, \forall a \in R$$
	\\Commutative Property: $$a+b=b+a,\forall a,b \in R$$
	\\So, the set of real numbers with usual addition is an abelian group.\\
	\\For multiplication:
	\begin{itemize}
	    \item Closure: $$\forall a, b \in R \Rightarrow a*b \in R$$
        \item Associative: $$a*(b*c)=(a*b)*c, \forall a,b,c \in R$$
	    \item Identity: For any $a \in R$, $1$ is an identity element of $R$ such that, $$a*1=1*a=a$$ .
		\item Distributive of $+$ over $*$:\\
	        $a*(b+c)=(a*b)+ (a*c)$  $ \forall a,b,c \in R $ \\
	        $(a+b)*c=(a*c)+(b*c)$   $ \forall a,b,c \in R $ \\
	   \item Commutative: $$a*b=b*a  \forall a,b \in R$$ \\
	   \item Multiplicative Inverse: $$a*a^{-1}= a^{-1} * a = 1 , \forall a^{-1} \in R$$\\
	    \\Therefore the set of real numbers with usual addition and multiplication is a field.
	 \end{itemize}
	\item The set of rational numbers with usual addition and multiplication
	\\ Solution:- 	\\Closure Property: $$a+b \in Q$$
	\\Associative Property: $$a+(b+c)=(a+b)+c, \forall a,b,c \in Q$$
	\\Identity Property: $$a+0=0+a=a, \forall a \in Q$$
	\\Inverse Property: $$ \ a + \ (-a\ ) =\ (-a\ ) + a = 0, \forall a \in Q$$
	\\Commutative Property: $$a+b=b+a,\forall a,b \in Q$$
	\\So, the set of rational numbers with usual addition is an abelian group.\\
	\\For multiplication:
	\begin{itemize}
	    \item Closure: $$\forall a, b \in Q \Rightarrow a*b \in Q$$
        \item Associative: $$a*(b*c)=(a*b)*c, \forall a,b,c \in Q$$
	    \item Identity: For any $a \in Q$, $1$ is an identity element of $Q$ such that, $$a*1=1*a=a$$ .
		\item Distributive of $+$ over $*$:\\
	        $a*(b+c)=(a*b)+ (a*c)$  $ \forall a,b,c \in Q $ \\
	        $(a+b)*c=(a*c)+(b*c)$   $ \forall a,b,c \in Q $ \\
	   \item Commutative: $$a*b=b*a  \forall a,b \in Q$$ \\
	   \item Multiplicative Inverse: $$a*a^{-1}= a^{-1} * a = 1 , \forall a^{-1} \in Q$$\\
	    \\Therefore the set of rational numbers with usual addition and multiplication is a field.
	 \end{itemize}
	\item The set of Even integers with usual addition and multiplication
		\\ Solution:- 	\\Closure Property: $$a+b \in 2N$$
	\\Associative Property: $$a+(b+c)=(a+b)+c, \forall a,b,c \in 2N$$
	\\Identity Property: $$a+0=0+a=a, \forall a \in 2N$$
	\\Inverse Property: $$ \ a + \ (-a\ ) =\ (-a\ ) + a = 0, \forall a \in 2N$$
	\\Commutative Property: $$a+b=b+a,\forall a,b \in 2N$$
	\\So, the set of even integers with usual addition is an abelian group.\\
	\\For multiplication:
	\begin{itemize}
	    \item Closure: $$\forall a, b \in R \Rightarrow a*b \in 2N$$
        \item Associative: $$a*(b*c)=(a*b)*c, \forall a,b,c \in 2N$$
	    \item Identity: For any $a \in Q$, $1$ is an identity element of $2N$ such that, $$a*1=1*a=a$$ .
		\item Distributive of $+$ over $*$:\\
	        $a*(b+c)=(a*b)+ (a*c)$  $ \forall a,b,c \in 2N $ \\
	        $(a+b)*c=(a*c)+(b*c)$   $ \forall a,b,c \in 2N $ \\
	    \item Commutative: $$a*b=b*a  \forall a,b \in 2N$$ \\
	    \item Multiplicative Inverse: $$a*a^{-1} = a^{-1} * a = 1 , \forall a^{-1} \notin 2N$$\\
	    \\Therefore the set of even integers with usual addition and multiplication is not a field.
    \end{itemize}	
\end{enumerate}
\end{ex}
\includegraphics{images/diagram1.png}
\subsection{Finite Fields of the form GF(p)}
\begin{itemize}
    \item Finite fields play a crucial role in many cryptographic algorithms.
    \item The order of a finite field (number of elements in the field) must be a power of a prime $p^{n} $ , where n is a positive integer.
    \item The finite field of order $p^{n} $ is generally written GF($p^{n} $); GF stands for Galois field.
    \item For a given prime, p, we define the finite field of order p, GF(p), as the set $Z_p$ of integers {0, 1, … , p - 1} together with the arithmetic operations modulo p. 
    \item Any integer in $Z_n$ has a multiplicative inverse if and only if that integer is relatively prime to n.
    \item If n is prime, then all of the nonzero integers in $Z_n$ are relatively prime to n, and therefore there exists a multiplicative inverse for all of the nonzero integers in $Z_n$.
    \item For example:- $Z_5$={0,1,2,3,4}
    \\1 mod 5 = 1; 1 is multiplicative inverse of itself
    \\(2*3) mod 5 = 1; 2 is multiplicative inverse of 3
    \\(3*2) mod 5 = 1; 3 is multiplicative inverse of 2
    \\(4*4)mod 5 = 1; 4 is multiplicative inverse of itself
    \\
    \item If a and b are relatively prime, then b has a multiplicative inverse modulo a. That is, if gcd(a, b) = 1, then b has a multiplicative inverse modulo a.
    \item Galois Field
    $$GF(P)=(Z_p,\oplus,\otimes)$$
    \\for $GF(2^{n})$, we use irreducible polynomial
\end{itemize}
\end{defn}
\section{Polynomial arithmetic}
\subsection{Ordinary Polynomial Arithmetic}
\begin{itemize}
    \item A polynomial of degree n (integer $n \geq 0$) is an expression of the form
    $$ $$ f(x)= $a_n X^{n} + a_{n-1} X^{n-1} + a_{n-2} X^{n-2}+...+a_0$= $\sum_{i=0} ^{n} a_{i} X^{i}$
    \item A zero-degree polynomial is called a constant polynomial and is simply an element of the set of coefficients. An nth-degree polynomial is said to be a monic polynomial if $a_n$=1.
\end{itemize}
\subsection{Polynomial Arithmetic with Coefficients in $Z_p$}
\begin{itemize}
    \item When polynomial arithmetic is performed on polynomials over a field, then division is possible.
    \item Note:- this does not mean that exact division is possible. Within a field, given two elements and , the quotient a/b is also an element of the field. However, given a ring R that is not a field, in general, division will result in both a quotient and a remainder; this is not exact division.
    \includegraphics{images/diagram2.png}
\end{itemize}
\subsection{Irreducible Polynomial}
A polynomial f(x) over a field is called irreducible polynomial if and only if f(x) cannot be expressed as a product of two polynomials both over and of degree lower than that of f(x). An irreducible polynomial is also called as prime polynomial.\\
\includegraphics{images/diagram3.png}
\subsection{Finding the Greatest Common Divisor}
We can extend the analogy between polynomial arithmetic over a field and integer
arithmetic by defining the greatest common divisor as follows.The polynomial
is said to be the greatest common divisor of a(x) and b(x)if the following are true.
\begin{enumerate}
    \item c(x) divides both a(x) and b(x).
    \item Any divisor of a(x) and b(x) is a divisor of c(x).
\end{enumerate}
We can adapt the Euclidean algorithm to compute the greatest common divisor of two polynomials.\\
\includegraphics{images/diagram4.png}
\end{document}

